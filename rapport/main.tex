\documentclass[french]{layout/Report}
\usepackage{layout/Report}
\usepackage{layout/ReportFrontPage}

\usepackage{pgf,tikz,textcomp}
\usepackage[section]{placeins}
\usepackage{romannum}
\usetikzlibrary{arrows}
\usepackage{mathtools}
\usepackage{array}
\usepackage{color}
\usepackage{graphicx}
\usepackage{todonotes}

\DeclarePairedDelimiter\abs{\lvert}{\rvert}%

\begin{document}

\pagenumbering{arabic}


\frontpage{Télécommande à 1 canal par infrarouge}
%-------------------------------------------------------------------------------
\section{Introduction}
Le but est de concevoir un système d'émission - réception infrarouge avec adressage
permettant d'enclencher et déclencher un relais optique.

%-------------------------------------------------------------------------------
\section{Structure générale et principe}

On a divisé le système en 5 étages avec des interfaces définies:

\begin{tabular}{l p{10cm}}
Générateur de signal:   & Générerateur de N pulses de durée de 100\si{\micro\second} espacés de 1\si{\milli\second} en salves espacées de 100\si{\milli\second}. \\
LED driver:             & Sortie de puissance qui drive la LED IR. \\
Récepteur:              & Récepteur IR avec amplification et filtrage. La sortie est le signal digital des pulses. \\
Décodeur:               & Circuit logique de décodage du nombre de pulses. La sortie est un pulse d'environ ? \si{\milli\second} pour chaque salve correcte reçue. \\
Sortie:                 & Circuit de détection d'interruption du signal avec commutation et  du relais optique.
\end{tabular}
\todo[inline]{todo: durée de sortie du décodeur}

\begin{figure}[h]
\centering
\vspace{5mm}
\includegraphics[width=\textwidth]{fig/IRemote_schema_structure}
% \includegraphics[width=0.8\textwidth]{fig/schema_bloc}
\caption{Schéma bloc du système avec signaux}
\label{fig:schema_bloc}
\vspace{5mm}
\end{figure}

Le but de la structure choisie est de faciliter le développement et la testabilité
du sous-système.



%-------------------------------------------------------------------------------
\section{Notation et nomenclature}

\begin{center}
    \begin{tabular}{| c | l |}
        \hline
        $n_{pulse}$  & Nombre de pulses dans une salve (addressage) \\ \hline
        $t_0$               & \\ \hline
    $T_0$               & \\ \hline
    $D_0$               & Duty cycle dans une salve \\ \hline
    $\tau$          & \\ \hline
    $t_{miss}$  & \\ \hline
        $t_{true}$  & Durée du pulse signal une salve correcte \\ \hline
    \end{tabular}
\end{center}

\begin{center}
    \begin{tabular}{| c | l | c |}
        \textit{pulse}      &  \\ \hline
    \end{tabular}
\end{center}
\todo[inline]{TODO: tableau pas fini}

\subsection{Générateur de signal}

\todo[inline]{dimensionnement de valeurs R et C}

\subsection{LED driver}

\subsection{Récepteur et filtrage}

\subsection{Décodeur}
Le décodeur à 3 fonctions:
\begin{itemize}
    \item{Le comptage des pulses assuré par le \emph{decounter}. Celui-ci signal en \emph{active high} si $n_{pulse}$ ont été reçus par la tension d'entrée depuis son dernier \emph{reset}.}
    \item{La détection de pulse manquant assuré par le \emph{missing pulse detector}. Celui-ci signal en \emph{active low} si la tension d'entrée est maintenue \emph{low} pendant au moins $t_{miss} = xxx \si{\milli\second}$ après la fin d'un pulse.}
    \item{La génération du signal sortant assuré par un délais et des portes logiques. Celui si génère en \emph{active high} un pulse de $t_{true} = xx\si{\milli\second}$ si la salve est \emph{correcte}.}
\end{itemize}

Une salve est considéré correcte si le \emph{missing pulse }
\begin{center}
    \begin{tabular}{| m{4cm} | m{4cm} | l | c |}
        \hline
    % \thead{\emph{Missing pulse detector}} & \thead{\emph{Decounter}} & \thead{Statu}} & \thead{Sortie}}\ \hline
    % \thead{\emph{Missing pulse detector}} & \thead{\emph{Decounter}} & \thead{Statu} & \thead{Sortie}\\ \hline
        low     & low       & Nombre incorrect de pulses dans la salve & low \\ \hline
        low     & high  & Salve correcte    & high\\ \hline
        high    & low       & Salve non finie & low \\ \hline
        high    & high  & Salve non finie & low \\ \hline
    \end{tabular}
\end{center}

\subsection{Sortie}

\section{Annexes}


\end{document}
